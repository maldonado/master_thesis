\pagenumbering{arabic}
\setcounter{page}{1}

\chapter{Introduction}

\section{Object Oriented Programming}

Object oriented concepts are not new to the field of computer
programming.  Almost all aspects of what we currently include under
the umbrella of the object oriented paradigm were present in early
programming languages.  Work on SIMULA \cite{Nygaard81} began in
Norway in the early 60's.  The first version was a simulation
language.  The final version was designed to be a general purpose
programming language.  By the time SIMULA 67 was finished, it
contained most of the features we commonly think of as being essential
to an object oriented programming language.  Other object oriented
programming languages did not start appearing until about the early
eighties, but they are becoming very popular.

Much of the motivation for the current popularity of object oriented
programming languages is a direct result of the high cost of software
creation and maintenance.  Maintenance can be described as modifying
software that has already been delivered.  These modifications can
consist of error corrections or alterations as a result of design or
specification changes.  On page 536 in \cite{Sommerville89},
Sommerville points out that ``\ldots maintenance costs are, by far,
the greatest cost incurred in developing and using a system.''

The rest of this section contains an introduction to some of the most
important features of the object oriented paradigm.  The final section
of this chapter is an introduction to the Dee System.

\subsection{Classes and Objects}

A {\em module} is a syntactic unit of a programming language.
Programs can be created by composing several modules, with each module
having its own set of variables and procedures.  The way in which
other syntactic units of a program can access the internals of a
module define its {\em interface}.  A {\em class} is a kind of module;
it is an abstraction of an entity that is manipulated by the software
system.  The difference between a class and a module lies in the type
of interface they provide.  In languages that support modules, a
module can often export types, constants, variables and procedures.
In most languages that support the object oriented paradigm, classes
can only\footnote{Usually, a class also implicitly exports one type so
that we can declare instances of the class.} export procedures that
can be performed on instances of the particular class exporting them.
In object oriented parlance, an instance of a class is called an {\em
object}.

The class is the main conduit used in the object oriented paradigm to
solve software cost problems.  Unlike the techniques used in
structured, top-down programming, system modularity using classes is
more closely related to data than to functionality.  When designing an
object oriented system, one first looks at the objects that are
manipulated by the system rather than the functionality of the system
\cite{Meyer88}. 

One example of an object manipulated by a system might be a stack.  A
particular system may have several stacks as data structures.  Each of
these stacks might be an object of the same class, that is, the class
{\tt Stack}.  The ways in which an object of the class {\tt Stack}
performs stack operations is by receiving a request to act from
another object.

\subsection{Messages and Interfaces}

A {\em method} is a procedure defined for a particular class.  A {\em
message} is a request by one object, to have another object execute
one of its methods.  This idea of exporting methods is related to the
concept of a message.  When a class exports a particular method it
means that an instance of the class will accept a message that
corresponds to a particular method.  When a message is received by an
object, it executes the appropriate method.  Only messages that have
been explicitly exported by the class may safely be received by
instances of that class.  It is the job of the type system and
semantic analyzer to ensure that such unexpected messages are never
sent.

Because in most object oriented programming languages data internal to
an object can not be exported, it can only be manipulated by methods
exported by its class.  In languages that do allow variables to be
exported, this usually\footnote{In some languages it is possible to
make exported variables writable.  The {\tt friends} specifier in C++
is an example.} makes them {\em read only} and saves the programmer
from having to write a function that simply returns a variable's
value. Exported methods may, in turn, invoke {\em private} (not
exported) methods of the class, which may also alter an object's data.
Still, in most cases, any change in the state of an object is
ultimately the result of a received message, corresponding to a public
method.  This results in a high degree of data encapsulation and
information hiding at the object level.

\subsection{Genericity}

Genericity permits the passing of type parameters to classes.  It is
similar to argument passing in functions.  This passing of types
allows one class to operate on many different types.  In
\cite{Meyer88} Meyer uses the example of a stack.  A portion of the
Eiffel code for class {\em STACK} with a generic type parameter is
shown in figure \ref{prog:STACK}

\begin{shortfigure}
\begin{verbatim}
-- Stack elements of an arbitrary type T
class STACK[T] export nb_elements, empty, push, pop, top
feature

    push ( x : T ) is
      -- Add x on top of the stack
    do ... end

    top : T is
      -- the element on the top of the stack
    do ... end

    pop : T is
      -- remove the top element from the stack and return it
    do ... end

    .
    .
    .

end -- class STACK
\end{verbatim}
\caption{Eiffel code for class STACK}
\label{prog:STACK}
\end{shortfigure}

In this example, {\tt T} is a formal generic parameter to the class {\em
STACK}.  Genericity allows us to use the same implementation of stack
on several different data types.  We can have a stack of integers, a
stack of floats or even a stack of arrays of integers.  Absence of
genericity might force us to write a separate implementation for each
data type.  Untyped object oriented languages do not require
genericity to achieve the same degree of code reuse, but do not reap
the benefits of strong type checking.


\subsection{Exception Handling}
\label{sec:except}

Exceptions are used to indicate unexpected or error conditions
occurring in a section of code.  Some examples of exceptions might be
an attempt to divide by zero or to open a file that does not exist.
Means to deal with these problems are built into several (not
necessarily object oriented) programming languages including Eiffel
\cite{Meyer88}, Ada \cite{Booch83}, Smalltalk \cite{Goldberg83} and
CLU \cite{Liskov81}.  A {\em signal} is the means used to indicate
that an exception condition has occured.  We use the term {\em raise}
to describe the act of altering the flow of control with a signal.
Usually, the method of dealing with an exception is to allow a
non-local exit from the code in which it occurred.  A {\em handler} is
a section of code that has made known its willingness to deal with
these singularities.  When an exception occurs, a signal is raised
and control is transferred to a handler.

An example used by Grogono in \cite{Grogono90} is that of matrix
inversion.  It is difficult for a function to determine if a matrix is
invertible without doing much of the calculation required to actually
do the inversion.  This makes it undesirable to check the precondition
that the matrix is invertible before attempting the operation.
Exceptions give us a way of neatly exiting the inversion function if
it is unable to complete its task.  In addition, an exception and
handling mechanism built into the language gives the programmer
greater flexibility in determining how and where a problem should be
handled.  This is accomplished by allowing exception handlers the
option of dealing with the problem or passing it on to an outer scope
of handlers.




\section{The Dee System}

Code reuse and efficient program maintenance are two important goals
of almost every large software system.  Although object oriented
programming has frequently been offered as a way of solving these
problems, the potential benefits of the object oriented paradigm have
not yet been realized in currently available products. The Dee System
is a complete software development environment created with the aim of
solving these problems.

In designing the Dee system, the following six principles laid a
foundation on which to build.

\begin{enumerate}

\item Information should not be duplicated.

\item All information related to a program entity should be in one
place.

\item The programmer should not have to provide information that the
compiler can easily infer and display.

\item Semantic analysis performed by the compiler should not be
complicated.  

\item A language that is intended for the development of production
quality software should be strongly typed.

\item The language should support the chosen programming paradigm
completely and consistently.

\end{enumerate}

I present these principles so that the reader who is familiar with other
object oriented programming languages will understand some of the ways
in which they differ from Dee. A deeper explanation of these design
principles can be found in \cite{Grogono90}.

These principles were chosen because it is hoped that they enable the
Dee system to satisfy its three primary goals.

\begin{enumerate}

\item a programming language that provides full support for the object
oriented paradigm;

\item a programming environment for the language that supports all phases
of software development; and

\item a library of classes that facilitate both coding at a high level of
abstraction and efficient object code.

\end{enumerate}

\subsection{The Dee Programming Language}

The focal point of the Dee Project is the Dee programming language
created by Peter Grogono.  It contains most of the trademark features
of the object oriented paradigm including multiple inheritance,
encapsulation, genericity, automatic garbage collection and exception
handling.  In addition it contains several features not previously
found in programming languages.  These additional features center on
the data base of class interfaces maintained by the compiler and will
be discussed below.


\begin{figure}[htb]
\begin{center}
\begin{picture}(440,280) \thicklines

% Framed text boxes

\put(120,260){\framebox(120,20){Editor}}

\put(120,180){\framebox(120,20){Scanner/Parser}}

\put(300,140){\framebox(120,20){Code Generator}}

\put(120,100){\framebox(120,20){Semantic Analyzer}}

%\put(280,100){\framebox( 80,20){Maker}}

\put(0,60){\framebox( 80,20){Browser}}

\put(120,60,){\framebox(120,20){CIM}}

\put(280,60){\framebox( 80,20){Linker}}

% Ovals with text

\put(180,230){\makebox(0,0){Dee Source}}
\put(180,230){\oval(80,20)}

\put(180,150){\makebox(0,0){AST}}
\put(180,150){\oval(80,20)}

\put(180,30){\makebox(0,0){CIDB}}
\put(180,30){\oval(80,20)}

\put(350,10){\makebox(0,0){C Source}}
\put(350,10){\oval(80,20)}

% Arrows

\put(180,260){\vector(0,-1){20}}   % Editor -> Dee Source
\put(180,220){\vector(0,-1){20}}   % Dee Source -> Scanner/Parser
\put(180,180){\vector(0,-1){20}}   % Scannar/Parser -> AST
\put(120,70){\vector(-1,0){40}}    % Browser <- DBMS
\put(220,150){\vector(1,0){80}}   % AST -> Code Gernerator
\put(370,140){\vector(0,-1){120}}  % Code Generator -> C Source
\put(240,70){\vector(1,0){40}}     % DBMS -> Linker
%\put(320,100){\vector(0,-1){20}}   % Maker -> Linker
\put(320,60){\vector(0,-1){40}}    % Linker -> C Source

\put(180,130){\vector(0,-1){10}}   % AST <-> Semantic Analyzer
\put(180,130){\vector(0,1){10}}

\put(180,90){\vector(0,-1){10}}    % Semantic Analyzer <-> DBMS
\put(180,90){\vector(0,1){10}}

\put(180,50){\vector(0,-1){10}}    % DBMS <-> Interfaces
\put(180,50){\vector(0,1){10}}

%\put(260,70){\line(0,1){40}}       % DBMS -> Maker
%\put(260,110){\vector(1,0){20}}

\end{picture}
\end{center}
\caption{Dee System Organization}
\label{fig:system}
\end{figure}


Figure \ref{fig:system} gives an overall picture of the Dee System and
illustrates the way in which its individual components (rectangles)
and data structures (ovals) interact.  I have written the
editor\footnote{Actually, I have created specialized editing features
for the already existing editor called Emacs.}, the scanner, the
parser and the code generator.  I also created much of the Dee
run-time system including the garbage collector and several base
classes.  In addition, I contributed a large amount of effort to the
semantic analyzer originally written by Wai Ming Wong.  I will now
give a brief explanation of the most important parts of the Dee
system.  In depth explanations of the portions that I have
contributed are presented in Chapter 2.

\subsection{The Compiler}

The Dee compiler can be broken down into several main components along
the lines found in most compilers.  There is a scanner, a parser, a
semantic analyzer and a code generator.  The scanner and parser create
an Abstract Syntax Tree (AST) and the semantic analyzer decorates it.
Finally, the code generator traverses it and outputs C code.  Unlike
other compilers, the Dee compiler has an additional component called
the Class Interface Manager (CIM).  This module acts as an interface
between the compiler and the database of class interfaces.  Whenever
the semantic analyzer needs information about the interface of an
ancestor or client class, it queries the CIM.  If no errors are
encountered during compilation, the interface to the class being
compiled is also written to the CIM.

\subsection{The Browser}

The browser exists in several forms, the first completed and the most
primitive of which is the {\tt ri} command.  The {\tt ri} command was
written to allow the programmers working on the Dee project to examine
the interface to particular classes stored in the CIDB by making
requests to the CIM.  It was created to work as a test and verification
tool rather than a programming aid.  Only after the Dee compiler was
first being tested did we realize it was quite useful to Dee
programmers as a browser.  It now serves as the basic tool with which
more sophisticated browsers have been built.

The {\tt ri} command outputs a description of the interface for the
class specified on the command line.  It has several options that
determine the amount of information to display.  Some of these options
provide the ability to specify only one attribute, other options add
information about how and where specific attributes are defined and
implemented.  Figure \ref{fig:ri} is an example of a portion of the
output of the {\tt ri} command on the class {\tt File}.

\begin{shortfigure}
\begin{verbatim}
class File

inherit Stream  
uses String Bool  
Ancestors Output Input Device Stream  

public var handler:Int 
   source class: Device

public var option:String 
   source class: Stream

public method close 
   From
   Defined By : Device
   Implement By : Device

public method eof  : Bool 
-- return ture if reach the end of file.
 
   Dee Instruction
   Defined By : File
   Implement By : File

public method open 
-- Function open a file with its' option
 
   Dee Instruction
   Defined By : Device
   Implement By : File

public method read (n:Int ) : String 
   Dee Instruction
   Defined By : Input
   Implement By : Input

public method write (buffer:String )
   Dee Instruction
   Defined By : Output
   Implement By : Output
\end{verbatim}
\caption{Sample Output from the {\tt ri} command.}
\label{fig:ri}
\end{shortfigure}

This primitive command has been combined with two more sophisticated
programs to provide better browsing.  The first is a Dee mode for the
Emacs editor. When editing a Dee program, Dee mode allows a user to
place the cursor on the name of a class and invoke the {\tt ri}
command on that class.  A new Emacs window is opened with the output
from {\tt ri} displayed in it.  Figure \ref{fig:dmode} shows an
example of the browser being invoked during an editing session.  See
Section \ref{dee-mode} for more details on Dee mode.

The second browser, written by Benjamin Cheung, runs under the X
window system and is called ``Dfolder''.  It is graphical, requires a
mouse, and provides a friendly user interface.  Options and classes
can be selected by pointing with the mouse and clicking the mouse
buttons.  Figure
\ref{fig:dfolder} is a snapshot of the Dfolder running on an X window
graphical display.  See \cite{Bcheung92} for more detailed information
on Dfolder.

\begin{shortfigure}
\centerline{\hbox{
\psfig{figure=dmode.ps}}}
\caption{An example of the Dee mode browser under Emacs.  The top 
window contains the program being edited.  The cursor is on the name
of the class being browsed in the lower window.}
\label{fig:dmode}
\end{shortfigure}

\begin{shortfigure}
\centerline{\hbox{
\psfig{figure=dfolder.ps}}}
\caption{An X window running the Dfolder browser.}
\label{fig:dfolder}
\end{shortfigure}

\subsection{The Editor}

The Dee system has an editor at the center of its user interface.  The
compiler cooperates with the editor in displaying and fixing syntax
and semantic errors.  As errors are discovered, the editor brings the
cursor to the delinquent line and allows the user to make a change
before moving on to the next error.  The editor also works closely
with the browser to select which classes to browse.

A future version of the Dee editor will be graphical and probably run
under the X window system.  The current version uses the Emacs editor,
which runs a special mode customized for Dee source programs.  See
Section \ref{dee-mode} for more details.


\subsection{The Linker}

The Dee linker generates a link file from a collection of classes,
starting from a given root class.  This file is eventually linked into
the final C linking phase which produces the Dee executable file. The
linker may also generate a Unix ``make'' file which will be used to
control the C compiler and linker.

Linking in Dee is a two step process; there is a Dee linking stage and
a C linking stage.  This extra linking stage, not normally found in
non-object oriented programming languages, is a result of dynamic
binding.  Because Dee allows inheritance, a variable at run-time can
be an instance of one of many different classes.  When a message is
sent to such an object, the correct method must be invoked.  The key
to choosing the correct method is figuring out which is the correct
slot in the method table.  The compiler can not determine the correct
slot because it is not until link time that the inheritance hierarchy
becomes fixed.  The same class, used in different programs, will have
a different place in the run-time inheritance graph.  The job of the
Dee linker is to arrange these tables so that, in cases where more
than one related class implements the same method, it is always at the
same slot in the method table.  The linker then assigns these offsets
to variables used at run-time to index the tables.  The linker uses
table compression techniques in an effort to minimize the size of
method tables.  More information about the linker and its
optimizations can be found in
\cite{Bcheung92} and \cite{Grogono90}.  An explanation of how code is
generated to allow dynamic binding at run-time can be found in Section
\ref{sec:applications} about applications.


\subsection{PC Dee}

Dee methods come in two flavors, regular and special.  {\em Regular}
methods are written in Dee.  {\em Special} methods are code that can
not be written in Dee, this has the added advantage of hiding
implementation details. Special methods are actually written in C.
For example, the steps needed to open a disk file for output are
dependent on particular operating systems.  In our Unix
implementation, the method {\tt open} in the class {\tt File} is a
special method.  It is written in C and uses the {\tt open()} system
call.  Dee programs can avoid knowing about such system dependences
because the class library hides these details.

The original implementation of Dee was written to run on a PC.  This
version of the language is different from the version currently
implemented on Unix in several ways.

The biggest difference is in the back end of the compiler.  The PC Dee
version emits code for a virtual Dee machine while the Unix version
emits C code.  The Dee machine is simulated by a big switch statement
with one label for each instruction in the virtual machine's
instruction set.  Eliminating the virtual machine in the Unix version
altered the syntax for the {\em special} Dee methods.  In the PC
version, a {\em special} method was like any other Dee method except
that it contained at least one virtual Dee machine instruction in its
body.  In the Unix version, because Dee machine instructions no longer
make sense, a {\em special} Dee method consists of a normal method
signature with its body replaced by the new keyword {\tt special}.
The Dee linker generates a standard C function prototype based on the
current class and the name of the method.  The programmer is
responsible for writing the C function to match the prototype.

The Unix version of the language has been slightly altered to avoid
using international characters.  This resulted in replacing the use of
the ``['' and ``]'' characters by ``('' and ``)'' for specifying
generic class parameters.  Comments were previously surrounded by
``\{'' and ``\}'' and are now begun with two dashes (``-- --'') and
terminated at the end of the line.  This convention was borrowed from
Ada because it has the advantage of never allowing run-on comments.
That is, if the programmer forgets to include the comment terminating
symbol (previously ``\}'') the parser can incorrectly interpret a
portion of the source code.  This usually leads to a plethora of
strange error messages being produced or a section of code not being
compiled.  With the Ada convention, it is almost impossible that the
programmer will forget the end of line character.

Despite these differences, PC Dee is close enough to the new version
to have been very helpful in making implementation decisions.

\section{The Advantages of Dee}

Dee is a true object oriented programming language. Thus Dee does not have the
complexity of a hybrid language such as Ada or C++.

Dee is a strongly-typed language. Thus Dee programs are easy to read and
do not fail at run-time. In this respect, Dee is more secure than untyped
languages such as CLOS and Smalltalk.

Dee provides ``semantic browsing.'' Semantic browsing depends on
features of the language and cannot be added retroactively to other
languages. The {\tt short} and {\tt flat} utilities of Eiffel
\cite{Meyer88} might at first sight seem to provide a capability
similar to the browsing features of Dee, but {\tt short} and {\tt
flat} read the source text of a class.  There is no guarantee that the
compiled version of the class is up-to-date, or even that the class
has been compiled. No information other than extracts from the source
text is provided. The Smalltalk browser is similarly limited.

Dee provides fully automatic storage management. In Ada and C++,
programmers are responsible for storage management. Errors in storage
allocation and release are common and notoriously hard to find and correct
\cite{Sakkinen88}. Storage management adds to both the size and complexity of
source code.

Class libraries are now available for Ada, C++, and Eiffel. All of them
provide data {\em structures\/}. The class library of Dee is different in
that it contains two levels of classes: data abstractions and data
structures. Data abstractions specify {\em what\/} tasks can be performed
and data structures describe {\em how\/} they are performed. Programmers
code using data abstractions and only later, when the program is almost
complete, choose appropriate data structures to represent the abstractions.


\section{Motivation For The Dee Project}

Reducing the high cost of software maintenance is one of primary
design objectives of Dee.  My work on the Dee project is ultimately
concerned with determining how well this goal was attained.  A fast
and efficient compiler is only as useful as the language being
compiled. An equally important result will be offering alternatives in
places where Dee falls short of this goal.

My motivation for working on the Dee project comes from two different
categories: issues involved in designing and programming in an object
oriented language, and issues involved in writing an object oriented
compiler and environment.  Because the this is the first Dee compiler
for Unix and because we have very limited resources, performance is
not a primary goal.  This is not to say that we paid no attention to
this issue, just that reasonable performance was required, exceptional
performance was not.

Despite the fact that I did not design Dee, I was present at many
discussions which helped shape some of the language design decisions.
Ultimately these decisions are judged on what happens when they are
effected by the compiler writers and put to the test by end users
writing applications.
