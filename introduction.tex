% -*- root: cuthesis_masters.tex -*-  

Software systems are everywhere. It does not matter anymore what is the domain in question, the truth is software will be present and it will be important. Virtually every business in every sector depends on software systems to aid in the development, production, marketing, and support of its products and services. 

The technological advance allowed unprecedented advance. Many innovations in the fields of robotic manufacturing, nanotechnologies, and human genetics research all have been possible due the aid provided by computers and software.

However, one major implication of the technological advance is that quality matters. The impact of software quality is huge, a faulty software can bankrupt companies and individuals, it can expose sensitive information or even threat lives. For example, in 2012 Knight Capital, a specialized firm in executing trades for retail brokers, took \$440 millions in cash losses due to an error in a trading algorithm~\cite{Register:trading_algorithm}. In 2014, air traffic controllers lost voice contact with 400 airplanes they were tracking over the southwestern United States when the main voice communications system shut down unexpectedly~\cite{IEEESpectrum:airtraffic}. The US National Institute of Standards and Technology (NIST) estimated that software faults and failures cost the US economy 59.5 billion dollars a year \cite{NIST:economic_impacts}. 

At the core of software systems are software developers. Their decisions and expertise is what makes the difference between faulty and reliable software. However, keep a high quality level software project is not easy, and it has been proven to be a rather complex achievement. During the development of a software project software developers have to deal with conflicting goals that require software to be delivered quickly, with high quality, and on budget. In practice, achieving all of these goals at the same time can be challenging, causing a tradeoff to be made. Often these tradeoff means shortcuts and the implementation of sub-optimal solutions. These workarounds allows the project to move faster at the cost of its increased maintenance effort in the future. This phenomena is called \emph{technical debt}. 

Prior work has shown that technical debt is widespread in the software domain, is unavoidable, and can have a negative impact on the quality of the software \cite{Lim2012Software}. Due to the importance of technical debt, a number of studies empirically examined technical debt and proposed techniques to enable its detection and management. Most approaches to identify technical debt is based on static source analysis tools, as described in greater detail in Chapter \ref{literature_review}. 

However, there are limitations to these approaches. First, static analysis tools depends on arbitrary metrics and thresholds to detect technical debt, and deriving appropriate threshold values is a challenging open problem that has attracted the attention and effort of several researchers ~\cite{Oliveira2014CSMR,Fontana2015WETSoM,Fontana2015EMSE}. As a matter of fact, the approaches based on source code analysis suffer from high false positive rates~\cite{Fontana2016SANER}. Second, static analysis tools requires the construction of Abstract Syntax Trees or other more advanced source code representations. For instance, some code smell detectors that also provide refactoring recommendations to resolve the detected code smells~\cite{Tsantalis2011TSE,Tsantalis2015TSE} generate computationally expensive program representation structures, such as program dependence graphs~\cite{Graf2010SCAM}, and method call graphs~\cite{Ali2012ECOOP} in order to match structural code smell patterns and compute metrics.

More recently, another venue was explored to enable identification of technical debt, the use of source code comments. Potdar and Shihab~\cite{Potdar2014ICSME} devised an approach of 62 comment patterns (i.e., words and phrases) that was used by software developers to indicate ``not quite right code''. The detection of technical debt through source code comments does not suffer of the same limitations that static analysis tools as it is a more lightweight process that does not depend of source code representations, and does not depend on thresholds of any kind as the developers themselves are admitting the debt. Due to its nature, technical debt found in source code comments is referred as \emph{\SATD}.

Thus far, the identification of \SATD provide researches a lot of open questions and research opportunities. For example, the comment patterns approach does not take into consideration the many different types of technical debt, it heavily relies on the manual classification of source code comments and it has no means to be evaluated in terms of precision and recall. Therefore this thesis take these research opportunities to pavement a solid way in the identification of technical debt through source code comments. 

\section{Research Hypothesis}

Prior research and our industrial experience lead us to the formation of our research hypothesis. We believe that:

\conclusionbox{The identification of technical debt through source code comments remains limited. We hypothesize that source code comments are mostly overlooked by researches because 1) it is written in natural language, and therefore, very complex to automatically analyze it; 2) Only a small percentage of source code comments present opinions that can lead to the identification of technical debt; 3) It is very difficult to evaluate an approach performance without knowing all the possible instances of technical debt contained in a project. We believe that source code comment approaches to identify technical debt are important as they processes the opinions of the developers. Moreover, we believe that \SATD approaches can be combined with source code analysis tools to help eliminate false positives.}

The goal of this thesis is to propose an approach that can effectively identify \SATD comments. The thesis is divided into two main chapters. The first part focus on the definition and general understanding of technical debt where we manually analyze source code comments to better understand the nature of \SATD. The second part proposes an approach to identify technical debt using Natural Language Processing (NLP)techniques.

\section{Thesis Overview}

We now give an overview of the work presented in this thesis.

\subsection{Chapter 2: Literature Review}

Chapter \ref{literature_review} details the definition of the technical debt metaphor and its expansion over time as more developers starts adopting the term to communicate suboptimal solutions. The review include summarized and concise information extracted from websites, blogs and research papers from the creation of the metaphor until present date. The chapter concludes with a critical evaluation of current limitations in the area and the challenges surrounds the technical debt landscape.

\subsection{Chapter 3: Analyzing Source Code Comments and Different Types of Self-Admitted Technical Debt}
    
In this chapter we examine source code comments from 5 open source projects to determine the different types of technical debt. First, we propose four simple filtering heuristics to eliminate comments that are not likely to contain technical debt. Filtering out irrelevant comments is very helpful as we can focus our attention to the more insightful comments. Second, we manually classify the remainder comments (i.e., more than 33K comments), and we find that \SATD can be classified into five main types - design debt, defect debt, documentation debt, requirement debt and test debt. The  most common type of self-admitted technical debt is design debt, making up between 42\% to 84\% of the classified comments.

\subsection{Chapter 4: Proposing an Approach to Automatically Identify Self-Admitted Technical Debt}

In this chapter, we present an approach to automatically identify design and requirement \SATD using Natural Language Processing (NLP). We study 10 open source projects. Using our approach we are able to accurately identify \SATD, we also discuss about the features (i.e., words) that are the best indicators of design and requirement debt. Lastly, as training data is the most crucial point to apply and expand our approach, we conduct a detailed analysis on the necessary quantity of training data to obtain satisfactory classification performances.

\section{Thesis Contributions}

The major contributions of this thesis are as follows:
\begin{itemize}

\item A concise review of the state of the art in technical debt. Such a review is very important to situate interested researchers as the use of the metaphor is becoming more popular each passing day.

\item We contribute a rich dataset of self-admitted technical making the data used in this study publicly available. To the best of our knowledge, there is not similar data available and we believe that the dataset will help future research in the area of \SATD providing the necessary means to evaluate and apply different approaches.

\item We propose an automatic, NLP-based, approach that outperforms the current state-of-the-art in the identify design and requirement \SATD. 

\item We investigate the amount of training data necessary to effectively identify technical debt through an empirical experiment, giving support to future enhancement and expansion of our approach, such as train datasets to detect \SATD comments in different programming languages or idioms.

\end{itemize}
