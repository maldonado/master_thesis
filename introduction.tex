% -*- root: cuthesis_masters.tex -*-  

At the core of any software system is the software team who develop it. Their decisions and expertise is what makes the difference between solid well built systems and brittle implementations that my lead to countless hours of patch work. Ensuring the high quality of a software project is not easy, on the contrary it has been proven to be quite a challenge, almost utupic~\cite{kruchten2012IEEE}. Developers often have to deal with conflicting goals while developing a software system. Software needs to delivered quickly, without defects and on budget, all of these must happen in a rapidly changing environment.

In practice what happens is that all these conflicting constrains force a tradeoff to be made by software developers~\cite{Cunningham1992WPM}. Often this tradeoff means shortcuts and workarounds that results sub-optimal solutions~\cite{Seaman2011,Kruchten2013IWMTD}. These workarounds allow the project to move faster at first, helping software developers achieve short-term goals at the expense of increased maintenance effort in the future. This phenomena is called \emph{technical debt} (TD). Acquiring careless technical debt can halt a software system to a full stop.

Unfortunately, prior research on technical debt has shown that technical debt is widespread, unavoidable, and therefore, needs to be properly managed to not have a negative impact on the quality of the software system~\cite{Lim2012Software}. The first step to manage technical debt is the identification of technical debt, which is the main focus of this thesis. 

A number of studies empirically examined technical debt and proposed techniques to enable its detection and management. Most approaches to identify technical debt are based on static source analysis tools, as described in more detail in Chapter \ref{literature_review}. However, there are limitations to these approaches. First, static analysis tools depends on arbitrary metrics and thresholds to detect technical debt, and deriving appropriate threshold values is a challenging open problem that has attracted the attention and effort of several researchers ~\cite{Oliveira2014CSMR,Fontana2015WETSoM,Fontana2015EMSE}. In addition, the approaches based on source code analysis suffer from high false positive rates~\cite{Fontana2016SANER}. Also, static analysis tools requires the construction of Abstract Syntax Trees or other more advanced source code representations. For instance, some code smell detectors that also provide refactoring recommendations to resolve the detected code smells~\cite{Tsantalis2011TSE,Tsantalis2015TSE} generate computationally expensive program representation structures, such as program dependence graphs~\cite{Graf2010SCAM}, and method call graphs~\cite{Ali2012ECOOP} in order to match structural code smell patterns and compute metrics.

More recently, another approach used source code comments to identify technical debt. Potdar and Shihab~\cite{Potdar2014ICSME} devised an approach, which uses 62 comment patterns (i.e., words and phrases) used by software developers to indicate ``not quite right code''. The detection of technical debt through source code comments does not suffer of the same limitations that static analysis tools. For example, it is a more lightweight process that does not depend of source code representations, and does not depend on thresholds of any kind as the developers themselves are admitting the debt. Due to its nature, technical debt found in source code comments is referred to as \emph{\SATD}.

However, the identification of \SATD poses many challenges. Since comments are written in natural language, it is very difficult to automatically analyze them, and therefore, comments were analyzed manually by reading through each of them to identify those that indicated self-admitted technical debt. This poses a serious treat to the scalability of this approach. In addition, the comment patterns approach does not take into consideration the many different types of technical debt and it has no means to be evaluated in terms of precision and recall. 

At the same time, automatic techniques that leverage machine learning have been proposed to help automatically classify natural language corpora, referred to as NLP techniques. In this thesis, we apply these NLP techniques to automatically identify technical debt from source code comments. The thesis provides two main contributions. The first part focuses on the definition and understanding of technical debt where we manually analyze source code comments to gain insights on the nature of \SATD. The second part proposes an approach to identify technical debt using Natural Language Processing (NLP) techniques.

\section{Research Hypothesis}

Prior research and our industrial experience lead us to the formation of our research hypothesis. We believe that:

\conclusionbox{The identification of technical debt remains limited, and mostly dependent of static source code analysis tools that require expensive and heavy analysis processes, while yielding too many false positives. We hypothesize that source code comments can improve the identification of technical debt. Thus far, the approaches that uses source code comments heavily depend on manual processes, which often do not scale well. We believe that Natural Language Processing (NLP) techniques, when provided with the appropriated training dataset, can tackle the current challenges in the identification of technical debt.}  

\section{Thesis Overview}


\noindent\textbf{Chapter 2: Literature Review:} In this chapter we discuss the definition of the technical debt metaphor and its expansion over time as more developers kept adopting the term to communicate suboptimal solutions and ``not quite right code''. The literature review provides summarized and concise information that was extracted from websites, blogs and research papers ranging from the creation of the metaphor to the present date in a chronological order. The chapter concludes with our critical evaluation of the current limitations in the field and the challenges surrounding technical debt.

\vspace{1mm}
\noindent\textbf{Chapter 3: Analyzing Source Code Comments and Different Types of Self-Admitted Technical Debt:} In this chapter we examine source code comments from 5 open source projects to determine the different types of technical debt. First, we propose four simple filtering heuristics to eliminate comments that are not likely to contain technical debt. Filtering out irrelevant comments is very helpful as it allows us focus our attention to the more insightful comments. Second, we manually classify the remaining comments (i.e., more than 33K comments), and we find that \SATD can be classified into five main types - design debt, defect debt, documentation debt, requirement debt and test debt. The two most common type of self-admitted technical debt are design and requirement debt, making up between 42\% to 84\% and 5\% to 45\% of the classified comments, respectively.

\vspace{1mm}
\noindent\textbf{Chapter 4: Proposing an Approach to Automatically Identify Self-Admitted Technical Debt:} In this chapter, we present an approach to automatically identify design and requirement \SATD using Natural Language Processing (NLP). We study 10 open source projects. We show that our approach can accurately identify \SATD, we also discuss the features (i.e., words) that are the best indicators of design and requirement debt. Lastly, as training data is the most crucial point to apply and expand our approach, we conduct a detailed analysis of the quantity of training data to obtain satisfactory classification performances.

\section{Thesis Contributions}

The major contributions of this thesis are as follows:
\begin{itemize}

\item A concise review of the state-of-the-art in the technical debt field. Such a review provide the necessary background to enable interested researchers to focus on the main challenges in the field of technical debt. 

\item We contribute a rich dataset of self-admitted technical making the data used in this thesis publicly available. To the best of our knowledge, there is no similar data available and we believe that the dataset will help future research in the area of \SATD providing the necessary means to evaluate and apply different approaches.

\item We propose an automatic, NLP-based, approach that outperforms the current state-of-the-art in the identifying design and requirement \SATD. Moreover, we investigate the amount of training data necessary to effectively identify technical debt through an empirical experiment, giving support to future enhancement and expansion of our approach, such as the detection of \SATD comments in different programming languages or idioms.

\end{itemize}